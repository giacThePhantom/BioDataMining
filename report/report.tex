%%%%%%%%%%%%%%%%%%%%%%%%%%%%%%%%%%%%%%%%%
% Stylish Article
% LaTeX Template
% Version 2.2 (2020-10-22)
%
% This template has been downloaded from:
% http://www.LaTeXTemplates.com
%
% Original author:
% Mathias Legrand (legrand.mathias@gmail.com)
% With extensive modifications by:
% Vel (vel@latextemplates.com)
%
% License:
% CC BY-NC-SA 3.0 (http://creativecommons.org/licenses/by-nc-sa/3.0/)
%
%%%%%%%%%%%%%%%%%%%%%%%%%%%%%%%%%%%%%%%%%

%----------------------------------------------------------------------------------------
%	PACKAGES AND OTHER DOCUMENT CONFIGURATIONS
%----------------------------------------------------------------------------------------

\documentclass[fleqn,10pt]{SelfArx} % Document font size and equations flushed left

\usepackage[english]{babel} % Specify a different language here - english by default
\usepackage{pgfgantt}

\usepackage{lipsum} % Required to insert dummy text. To be removed otherwise

%----------------------------------------------------------------------------------------
%	COLUMNS
%----------------------------------------------------------------------------------------

\setlength{\columnsep}{0.55cm} % Distance between the two columns of text
\setlength{\fboxrule}{0.75pt} % Width of the border around the abstract

%----------------------------------------------------------------------------------------
%	COLORS
%----------------------------------------------------------------------------------------

\definecolor{color1}{RGB}{0,0,90} % Color of the article title and sections
\definecolor{color2}{RGB}{0,20,20} % Color of the boxes behind the abstract and headings

%----------------------------------------------------------------------------------------
%	HYPERLINKS
%----------------------------------------------------------------------------------------

\usepackage{hyperref} % Required for hyperlinks

\usepackage{float}
\usepackage{makecell}
\hypersetup{
	hidelinks,
	colorlinks,
	breaklinks=true,
	urlcolor=color2,
	citecolor=color1,
	linkcolor=color1,
	bookmarksopen=false,
	pdftitle={Title},
	pdfauthor={Author},
}

%----------------------------------------------------------------------------------------
%	ARTICLE INFORMATION
%----------------------------------------------------------------------------------------

\JournalInfo{Laboratory of biological data mining} % Journal information
\Archive{Project report} % Additional notes (e.g. copyright, DOI, review/research article)

\PaperTitle{Identification and validation of a vitamin D-related prognostic signature in colorectal cancer} % Article title

\Authors{Diego Barquero Morera\textsuperscript{1}, Giacomo Fantoni\textsuperscript{2}, Gaia Faggin\textsuperscript{3}, Leonardo Golinelli\textsuperscript{4}} % Authors
\affiliation{\textsuperscript{1}\textit{diego.barqueromorera@studenti.unitn.it}} % Author affiliation
\affiliation{\textsuperscript{2}\textit{giacomo.fantoni@studenti.unitn.it}} % Author affiliation
\affiliation{\textsuperscript{3}\textit{gaia.faggin@studenti.unitn.it}} % Author affiliation
\affiliation{\textsuperscript{4}\textit{leonardo.golinelli@studenti.unitn.it}} % Author affiliation

\Keywords{} % Keywords - if you don't want any simply remove all the text between the curly brackets
\newcommand{\keywordname}{Keywords} % Defines the keywords heading name

%----------------------------------------------------------------------------------------
%	ABSTRACT
%----------------------------------------------------------------------------------------

\Abstract{}
%----------------------------------------------------------------------------------------

\begin{document}

\maketitle % Output the title and abstract box

\tableofcontents % Output the contents section

\thispagestyle{empty} % Removes page numbering from the first page

%----------------------------------------------------------------------------------------
%	ARTICLE CONTENTS
%----------------------------------------------------------------------------------------

\section{Introduction}

\section{Material and methods}

	\subsection{Data preprocessing}
	This project requires a lot of data to achieve high level of statistica significance, so we decided to start with an high number of samples, taken from different dataset.
	A brief description of each dataset considered for the pipeline can be found in table \ref{tab:datasets}.

	\begin{table*}[ht]
		\centering
		\begin{tabular}{cccc}
			\hline
			Dataset name & Sample description & Number of samples\\
			\hline
			E-MTAB-6698	& healthy and tumor colorectal samples	&1566\\
			GSE157982	&baseline and vit. D-treated CRC rectal samples	&98\\
			GSE38832	&tumor colorectal samples	&122\\
			TCGA-COAD	&tumor colorectal samples	&438\\
			GSE14333	&tumor colorectal samples	&290\\
			GSE17536	&tumor colorectal samples	&177\\
			GSE31595	&tumor colorectal samples	&37	\\
			GSE33113	&tumor colorectal samples	&96	\\
			GSE38832	&tumor colorectal samples	&122\\
			GSE39084	&tumor colorectal samples	&70	\\
			GSE39582	&tumor colorectal samples	&585\\
			GSE103479	&tumor colorectal samples	&156\\
			GSE17537	&tumor colorectal samples	&55	\\
			\hline
		\end{tabular}
		\caption{Starting datasets}
		\label{tab:datasets}
	\end{table*}

		\subsubsection{Sample splitting and filtering}
		The first step of the project's pipeline is to filter the datasets in order to obtain a list of the samples having all the data necessary for the downstream analyses.
		So, analysing all of the datasets' metadata we split all the samples in $6$ sets according to which clinical data was available.
		After this operation we retained $2676$ samples out of the starting $3812$ ($70\%$) divided as in table \ref{tab:samples_split}.

		\begin{table}[H]
			\centering
			\begin{tabular}{cc}
				\hline
				Set usage & $n^\circ$ of samples\\
				COX fitting & $388$\\
				KM curve & $157$\\
				Vitamine D low & $49$\\
				Vitamine D high & $49$\\
				Stage low & $1120$\\
				Stage high & $908$\\
				\hline
			\end{tabular}
			\caption{Split samples}
			\label{tab:samples_split}
		\end{table}

		\subsubsection{Dataset normalization}

	\subsection{Differentially expressed genes}

		\subsubsection{DEG between stage of cancer}

		\subsubsection{Vitamin D gene signature}

		\subsubsection{Pathway enrichment}

	\subsection{Survival analysis}

\section{Results}

	\subsection{Data preprocessing}

		\subsubsection{Sample splitting and filtering}

		\subsubsection{Dataset normalization}

	\subsection{Differentially expressed genes}

		\subsubsection{DEG between stage of cancer}

		\subsubsection{Vitamin D gene signature}

		\subsubsection{Pathway enrichment}

	\subsection{Survival analysis}


\section{Discussion}

%----------------------------------------------------------------------------------------
%	REFERENCE LIST
%----------------------------------------------------------------------------------------
\phantomsection
\bibliographystyle{unsrt}
\bibliography{references}


\end{document}
